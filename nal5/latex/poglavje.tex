\section{Rešitev}

Najprej si poglejmo avtokorelacijo za obe sovi.
\begin{figure}[h]
    \centering
    \includegraphics[width=12cm]{pdfs/bubomono.txt_acor.pdf}
    \vspace{10pt}
    \includegraphics[width=12cm]{pdfs/bubo2mono.txt_acor.pdf}
    \caption{Avtokorelacija signala sov}
\end{figure}

Preden sem koreliral signal sem vektorja samplov normiral, tako da je
"najmočnejši" signal 1. To pomeni \[\|x\|_\infty = \max_i |x_i|\]

Poglejmo si še korelacije med sovami in posnetki:
\newpage
\begin{figure}[h]
    \centering
    \foreach \mix in {mix, mix1, mix2, mix22} {
        \foreach \sova in {bubomono, bubo2mono}{
            \includegraphics[width=8cm]{pdfs/cor_\mix.txt_\sova.txt.pdf}
        }
    }
    \caption{Korelacija sov in posnetkov iz narave}
\end{figure}
Tukaj je bila normalizacija drugače izbrana. Tukaj je bila narejena normalizacija korelacije. Če je varianca signala a $\sigma_a$,
potem je $ \mathbf{x} = \mathbf{x} / \left(\sigma_a \sigma_b |\mathbf{b}|\right)$. Sigma je izračunana z \verb|numpy.std()|.


Hitrosti so precej dolgčasno pričakovane ampak vseeno.
\begin{figure}[h]
    \centering
    \includegraphics[width=8cm]{pdfs/cas-lin.pdf}
    \vspace{10pt}
    \includegraphics[width=8cm]{pdfs/cas-log.pdf}
    \caption{Hitrost algoritmov}
\end{figure}
\newpage
\section{Dodatna naloga}
Posnel sem dva človeka m in ž ko izgovarjata aaaaaaaaaaaa. Posnel sem jih tudi
ko bereta slovar. Gledal sem ali lahko spet zaznam ali gre za ž glas ali m glas.
Poglejmo si spektra njunega glasu.
\begin{figure}[h]
    \begin{center}
        \includegraphics[width=12cm]{pdfs/fft_spol.pdf}
    \end{center}
    \caption{Barva glasu moškega in ženske}
\end{figure}

% Glasova sta bila prej normirana, saj je bil moški glas posnet bližje mikrofona.
% Poglejmo si zopet korelacijo med posnetki:
% \foreach \gender in {Z, M} {
%     \foreach \mix in {Glas\ 001, Glas\ 002}{
%         pdfs/cor_\gender_\mix.pdf 
%     }
% }
\begin{figure}[h]
    \begin{center}
        
    \foreach \gender in {Z, M} {%
        \foreach \mix in {001, 002} {%
            \includegraphics[width=8cm]{pdfs/cor_\gender_Glas\space\mix.pdf}%    
        }%
        \par
}%
    \end{center}
    
    \caption{Korelacija glasa človeka in posnetkov iz branja slovarja}
\end{figure}