\documentclass[slovene,11pt,a4paper]{article}
\usepackage[margin=2cm,bottom=3cm,foot=1.5cm]{geometry}
\setlength{\parindent}{0pt}
\setlength{\parskip}{0.5ex}

\usepackage[pdftex]{graphicx}


\usepackage{a4wide} %najaci package
\usepackage[utf8]{inputenc}
\usepackage[slovene]{babel}
\usepackage{color}
\usepackage{graphicx}
\usepackage{subfigure}
\usepackage{imakeidx}
\usepackage{adjustbox}
\usepackage{float}
\usepackage{amsmath}
\usepackage{mathtools}
\usepackage{tikz}
\usepackage{amssymb}
\usepackage{listings}
\usepackage{siunitx}
\usepackage{hyperref}
\usepackage{amsfonts}
\usepackage{mathrsfs}

\def\phi{\varphi}
\def\eps{\varepsilon}
\def\theta{\vartheta}

\newcommand{\thisyear}{2024/25}

\renewcommand{\Re}{\mathop{\rm Re}\nolimits}
\renewcommand{\Im}{\mathop{\rm Im}\nolimits}
\newcommand{\Tr}{\mathop{\rm Tr}\nolimits}
\newcommand{\diag}{\mathop{\rm diag}\nolimits}
\newcommand{\dd}{\,\mathrm{d}}
\newcommand{\ddd}{\mathrm{d}}
\newcommand{\ii}{\mathrm{i}}
\newcommand{\lag}{\mathcal{L}\!}
\newcommand{\ham}{\mathcal{H}\!}
\newcommand{\four}[1]{\mathcal{F}\!\left(#1\right)}
\newcommand{\bigO}[1]{\mathcal{O}\!\left(#1\right)}
\newcommand{\sh}{\mathop{\rm sinh}\nolimits}
\newcommand{\ch}{\mathop{\rm cosh}\nolimits}
\renewcommand{\th}{\mathop{\rm tanh}\nolimits}
\newcommand{\erf}{\mathop{\rm erf}\nolimits}
\newcommand{\erfc}{\mathop{\rm erfc}\nolimits}
\newcommand{\sinc}{\mathop{\rm sinc}\nolimits}
\newcommand{\rect}{\mathop{\rm rect}\nolimits}
\newcommand{\ee}[1]{\cdot 10^{#1}}
\newcommand{\inv}[1]{\left(#1\right)^{-1}}
\newcommand{\invf}[1]{\frac{1}{#1}}
\newcommand{\sqr}[1]{\left(#1\right)^2}
\newcommand{\half}{\frac{1}{2}}
\newcommand{\thalf}{\tfrac{1}{2}}
\newcommand{\pd}{\partial}
\newcommand{\Dd}[3][{}]{\frac{\ddd^{#1} #2}{\ddd #3^{#1}}}
\newcommand{\Pd}[3][{}]{\frac{\pd^{#1} #2}{\pd #3^{#1}}}
\newcommand{\avg}[1]{\left\langle#1\right\rangle}
\newcommand{\norm}[1]{\left\Vert #1 \right\Vert}
\newcommand{\braket}[2]{\left\langle #1 \vert#2 \right\rangle}
\newcommand{\obraket}[3]{\left\langle #1 \vert #2 \vert #3 \right \rangle}
\newcommand{\hex}[1]{\texttt{0x#1}}

\renewcommand{\iint}{\mathop{\int\mkern-13mu\int}}
\renewcommand{\iiint}{\mathop{\int\mkern-13mu\int\mkern-13mu\int}}
\newcommand{\oiint}{\mathop{{\int\mkern-15mu\int}\mkern-21mu\raisebox{0.3ex}{$\bigcirc$}}}

\newcommand{\wunderbrace}[2]{\vphantom{#1}\smash{\underbrace{#1}_{#2}}}

\renewcommand{\vec}[1]{\overset{\smash{\hbox{\raise -0.42ex\hbox{$\scriptscriptstyle\rightharpoonup$}}}}{#1}}
\newcommand{\bec}[1]{\mathbf{#1}}


\title{
Matematično-fizikalni praktikum\\
\bigskip
\bf\Large 3.~naloga: Lastne vrednosti in lastni vektorji
}

\author{Tadej Tomažič}

\makeindex[columns=3, title=Alphabetical Index, intoc]


\newcommand{\bi}[1]{\hbox{\boldmath{$#1$}}}
\newcommand{\bm}[1]{\hbox{\underline{$#1$}}}

\begin{document}
\pagenumbering{gobble} 
\author{Tadej Tomažič}
\date{\today}

\maketitle

\newpage
\pagenumbering{arabic}
\tableofcontents
\listoffigures
\newpage


\section{Naloga}
Enodimenzionalni linearni harmonski oscilator (delec mase $m$
s kinetično energijo $T(p)=p^2/2m$ v kvadratičnem potencialu
$V(q)=m\omega^2 q^2/2$) opišemo z brezdimenzijsko Hamiltonovo funkcijo
\begin{equation*}
  H_0 = {1\over 2} \left( p^2 + q^2 \right) \>,
\end{equation*}
tako da energijo merimo v enotah $\hbar\omega$, gibalne količine
v enotah $(\hbar m\omega)^{1/2}$ in dolžine v enotah $(\hbar/m\omega)^{1/2}$.
Lastna stanja $|n\rangle$ nemotenega Hamiltonovega operatorja $H_0$
poznamo iz osnovnega tečaja kvantne mehanike [Strnad III]:
v koordinatni reprezentaciji so lastne valovne funkcije
\begin{equation*}
  |n\rangle = (2^n n! \sqrt{\pi})^{-1/2} \mathrm{e}^{-q^2/2}\,  {\cal H}_n (q)\>,
\end{equation*}
kjer so ${\cal H}_n$ Hermitovi polinomi.
Lastne funkcije zadoščajo stacionarni Schr\"odingerjevi enačbi
\begin{equation*}
H_0 | n^0 \rangle = E_n^0 | n^0 \rangle
\end{equation*}
z nedegeneriranimi lastnimi energijami $E_n^0 = n + 1/2$
za $n=0,1,2,\ldots~$.  Matrika $\langle i | H_0 | j\rangle$
z $i,j=0,1,2,\ldots,N-1$ je očitno diagonalna, z vrednostmi
$\delta_{ij}(i + 1/2)$ po diagonali.  Nemoteni Hamiltonki
dodamo anharmonski člen
\begin{equation*}
H = H_0 + \lambda q^4 \>.
\end{equation*}
Kako se zaradi te motnje spremenijo lastne energije?
Iščemo torej matrične elemente $\langle i | H | j\rangle$
{\sl motenega\/} Hamiltonovega operatorja v bazi {\sl nemotenih\/}
valovnih funkcij $| n^0\rangle$, kar vemo iz perturbacijske
teorije v najnižjem redu.  Pri računu si pomagamo
s pričakovano vrednostjo prehodnega matričnega
elementa za posplošeno koordinato
$$
q_{ij} = \langle i | q | j \rangle
       = {1\over 2} \sqrt{i+j+1}\,\, \delta_{|i-j|,1} \>,
$$
ki, mimogrede, uteleša izbirno pravilo za električni dipolni
prehod med nivoji harmonskega oscilatorja.  V praktičnem računu
moramo seveda matriki $q_{ij}$ in $\langle i | H | j\rangle$
omejiti na neko končno razsežnost $N$.

\bigskip

{\sl Naloga\/}: Z diagonalizacijo poišči nekaj najnižjih lastnih
vrednosti in lastnih valovnih funkcij za moteno Hamiltonko
$H = H_0 + \lambda q^4$
ob vrednostih parametra $0\le\lambda\le 1$.  Rešujemo torej
matrični problem lastnih vrednosti
\begin{equation*}
  H | n \rangle = E_n | n \rangle \>.
\end{equation*}
Nove (popravljene) valovne funkcije $| n\rangle$ so seveda
linearna kombinacija starih (nemotenih) valovnih funkcij $| n^0\rangle$.
Matrike velikosti do $N=3$ ali $N=4$ lahko za silo diagonaliziramo peš;
za diagonalizacijo pri večjih $N$ uporabi enega ali več numeričnih postopkov,
na primer rutine {\tt tred2} in {\tt tqli}
iz zbirke Numerical Recipes ali iz kakega drugega vira (npr Python). Vsaj enega izmed
postopkov izvedi 'ročno' (sprogramiraj, uporabi izvorno kodo).  Preveri,
da v limiti $\lambda\to 0$ velja $E_n\to E_n^0$
(če ne velja, je verjetno nekaj narobe s programom).
Razišči, kako so rezultati odvisni od razsežnosti $N$ matrik
$H_0$ oziroma $q^4$.  Kakšna je konvergenca lastnih vrednosti
pri velikih $N$?

\bigskip

Namesto da računamo matrične elemente
$q_{ij}=\langle i | q | j \rangle$ in perturbacijsko matriko
razumemo kot $[ q_{ij} ]^4$, bi lahko računali tudi matrične
elemente kvadrata koordinate
\begin{equation*}
q^{(2)}_{ij} = \langle i | q^2 | j \rangle
\end{equation*}
in motnjo razumeli kot kvadrat ustrezne matrike,
\begin{equation*}
\lambda q^4 \to \lambda \left[ \, q^{(2)}_{ij} \,\right]^2 \>,
\end{equation*}
ali pa bi računali matrične elemente četrte potence koordinate
\begin{equation*}
q^{(4)}_{ij} = \langle i | q^4 | j \rangle
\end{equation*}
in kar to matriko razumeli kot motnjo,
\begin{equation*}
\lambda q^4 \to \lambda \left[ \, q^{(4)}_{ij} \,\right] \>.
\end{equation*}
Kakšne so razlike med naštetimi tremi načini izračuna
lastnih vrednosti in funkcij?  Pri računu poleg enačbe za $\langle i | q | j \rangle$ uporabi še enačbi
\begin{equation*}
\langle i|q^2|j\rangle
  = {1\over 2} \biggl[
    {\sqrt{j(j-1)}} \, \delta_{i,j-2}
  + {(2j+1)} \, \delta_{i,j}
  + {\sqrt{(j+1)(j+2)}} \, \delta_{i,j+2} \biggr]
\end{equation*}
ter
\begin{eqnarray*}
\langle i|q^4|j\rangle
  = {1\over 2^4}\sqrt{2^i \, i!\over 2^{j} \, j! } \, \biggl[ \,
  &\,& \delta_{i,j+4} + 4\left(2j+3\right) \delta_{i,j+2}
                      + 12 \left(2j^2+2j+1\right) \, \delta_{i,j} \\[3pt]
  &+& 16j \left(2j^2-3j+1\right) \, \delta_{i,j-2}
     + 16j\left(j^3-6j^2+11j-6\right) \, \delta_{i,j-4} \biggr] \>,
\end{eqnarray*}
ki ju ni težko izpeljati iz rekurzijskih zvez za Hermitove
polinome.

\bigskip

{\sl Dodatna naloga\/}: Poišči še nekaj najnižjih lastnih energij
in lastnih funkcij za problem v potencialu z dvema minimumoma
\begin{equation*}
H = {p^2\over 2} - 2q^2 + {q^4\over 10} \>.
\end{equation*}
\section{Rešitev}
Test123 \inf
\end{document}
